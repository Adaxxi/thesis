% !Mode:: "TeX:UTF-8"

\titlecontents{chapter}[2em]{\vspace{.5\baselineskip}\xiaosan\song}
             {\prechaptername\CJKnumber{\thecontentslabel}\postchaptername\qquad}{} 
             {}                            % 设置该选项为空是为了不让目录中显示页码
\fancypagestyle{plain}   % 设置页眉页脚风格,按照教务处规定,此处出现页眉,但是没有页脚(页码)。
\lhead{}
\rhead{}
\chead{\song\wuhao 天津大学 2018 届本科生毕业设计(论文)} % 设置页眉内容
\lfoot{}
\cfoot{}
\rfoot{}
\markboth{致\quad 谢}{致\quad 谢}
\addcontentsline{toc}{chapter}{致\quad 谢} % 添加到目录中
\chapter*{致\quad 谢}
\setcounter{page}{1}
时间如白驹过隙,一转眼,四年紧张而又充实的大学生活即将画上句号。经过大半年时间的磨砺,毕业论文最终完稿,这一段时间学习论文、构思方法、实现并优化模型、整理思路直至完成论文,我得到了许多关怀和帮助,在此要向大家表达我诚挚的谢意。回首学习生活,对那些引导、激励我的老师同学们,我心中充满了感激。

首先要感谢的就是我的指导老师张鹏副教授,从选题、开题以及定稿,老师都倾注了极大的关心和和鼓励,敦促了我按质按量完成任务,老师还在百忙之中提出了许多中肯建议,使我在研究的过程中不至于迷失方向。老师渊博的学识、细致严谨的研究态度、求真务实的工作作风以及开放式的创新理念都给我留下来深刻印象,让我感慨良多,受益匪浅。

由于我在校外做毕设,感谢计算所的徐君老师,非常感谢他百忙之中还给予了我思路上的整体指导。同时还要感谢实验室的曾玮师姐在生活上给我提供了非常多关怀,何逸轩师兄在科研上给我提供了细致、全面的帮助,感谢于思皓师兄对我进行了许多思维训练。感谢实验室的每一位小伙伴,在学习和生活上对我的帮助。

毕业论文完成之际,也就意味着我要离开天津大学这个乐园,在此要感谢我生活学习了四年的母校,母校给了我一个宽阔的学习平台,让我不断吸取新知,充实自己。迈向更广阔的天地。同事也要感谢本科阶段的所有老师,你们辛苦了!还有所有支持和帮助过我的人,希望大家都能幸福平安!

最后,谨向百忙之中抽出时间参加答辩和审阅论文的每位评委老师表达诚挚的谢意!


