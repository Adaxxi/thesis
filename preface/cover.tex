% !Mode:: "TeX:UTF-8"

%%  可通过增加或减少 setup/format.tex中的
%%  第274行 \setlength{\@title@width}{8cm}中 8cm 这个参数来 控制封面中下划线的长度。

\cheading{天津大学~2018~届本科生毕业设计(论文)}      % 设置正文的页眉,需要填上对应的毕业年份
\ctitle{基于强化学习的文本匹配}    % 封面用论文标题,自己可手动断行
\caffil{计算机科学与技术学院} % 学院名称
\csubject{计算机科学与技术学院}   % 专业名称
\cgrade{2014级}            % 年级
\cauthor{陶舒畅}            % 学生姓名
\cnumber{3014216073}        % 学生学号
\csupervisor{张鹏副教授}        % 导师姓名
\crank{副教授}              % 导师职称

\cdate{\the\year~年~\the\month~月~\the\day~日}

\cabstract{
互联网飞速发展,使信息量呈爆炸式增长。文本匹配作为信息检索和冗余文本消除的基础任务,一直受到学术界和工业界的高度重视。许多任务如问答系统、机器翻译等都可归结成文本匹配问题。本文主要研究内容包括:

(1) 针对文本匹配问题,设计了马尔可夫决策过程中的状态、动作和奖励函数,实现了基于价值迭代算法的文本匹配模型,能够生成匹配路径。与经典的文本匹配模型的结果进行比较,结果表明基于值迭代算法的文本匹配模型,在各个评价准则下均优于其他经典文本匹配算法。

(2) 基于值迭代的文本匹配模型是贪心的方法,可能会出现局部最优解。本文设计了基于蒙特卡罗树搜索算法的文本匹配模型,并与其他经典算法进行对比。实验结果表明,基于蒙特卡罗树搜索的文本匹配模型具有显著的优势。

本文提出的两种基于强化学习的文本匹配模型都可以很好的完成该任务,生成匹配路径具有合理性。实验证明在真实数据集下,效果得到显著提升,同时收敛速度更快。
}

\ckeywords{文本匹配;强化学习;马尔科夫决策过程;蒙特卡洛树搜索}

\eabstract{
With the rapid development of the Internet, the amount of information has exploded. Text matching, as the basic task of information retrieval and redundant text elimination, has been highly valued by the academic community and industry. Many tasks such as question answering system, machine translation, etc. can be attributed to text matching problems. The main research contents of this article include:


(1) For text matching problem, this paper designs state, action and reward function in Markov decision process. We implement a text matching model based on value iterative algorithm. We compare the results of a real dataset training model with a classical deep learning text matching model. The experimental results show that the text matching model based on the value iterative algorithm is superior to other classic text matching algorithms under various evaluation criteria.

(2)The value-iteration-based text matching model is based on a greedy method. Local optimal solutions may appear for the linguistic compositional structure problem. This paper proposes a text matching model based on Monte Carlo search algorithm and compares it with other classical algorithms. The results show that the text matching model based on Monte Carlo search outperforms others.

The two text matching models based on reinforcement learning proposed in this paper can accomplish this task well. It is reasonable to generate matching paths. Experiments show that under the real data set, the effect is significantly improved and the convergence speed is faster.


}

\ekeywords{Text Matching, Reinforcement learning, Markov Decision Process, Monte Carlo Tree Search}

\makecover

\clearpage
